\documentclass[times, twoside, watermark]{zHenriquesLab-StyleBioRxiv}
\usepackage{blindtext}

% Please give the surname of the lead author for the running footer
\leadauthor{Henriques} 

\begin{document}

\title{Discovering vulnerabilities in Python web applications}
\shorttitle{My Template}

% Use letters for affiliations, numbers to show equal authorship (if applicable) and to indicate the corresponding author
\author[1,\Letter]{83531 - Miguel Belém}
\author[2,\Letter]{87701 - Ricardo Ferreira}
\author[3,\Letter]{83576 - Vítor Nunes}

\maketitle

%TC:break Abstract
%the command above serves to have a word count for the abstract
\begin{abstract}
\blindtext
\end {abstract}
%TC:break main
%the command above serves to have a word count for the abstract

\begin{keywords}
bla | bla | bla | bla
\end{keywords}

\begin{corrauthor}

\end{corrauthor}

\section*{Implementation}
Our tool was develop in Python 3.

We used a AST object tree structure as an internal representation for the input program. Therefore
each instruction is converted to an object. We have objects for all major instructions like \textsc{If},
\textsc{Assignment}, \textsc{While}, \textsc{Variable}, \textsc{Compare}, etc... each one containing
only interesting values gathered from the AST json tree.
In the end the idea is to construct one single object, called \textsc{Root} which contains a block of instructions.
Intuitively, each instruction could contain others. An \textsc{If} contains a condition, a body of instructions
in case of conditions evaluates to True, and a else body of instructions, otherwise. A slice is then represented
as a matryoska of objects. We opted for this representation due to its flexibility. This is
accomplish by a recursive function that constructs the objects. It could be possible to use an iterative
function instead, however we would loose flexibility and we would increase code complexity.

\subsection*{Explicit Flows}
%explain how we use a symtable

\subsection*{Implicit Flows}
%explain what we do to check implicit flows, like propagate the sources of a condition to every assigment
%in the body


\section*{Imprecisions and Limitiations}

\section*{Challenges}

\section*{Bonus}

\section*{Conclusions}

blablaba \ref{fig:computerNo} 
\blindtext

\section*{Bibliography}
\bibliography{zHenriquesLab-Mendeley}

%% You can use these special %TC: tags to ignore certain parts of the text.
%TC:ignore
%the command above ignores this section for word count
\onecolumn
\newpage

\section*{Word Counts}
This section is \textit{not} included in the word count. 
\subsection*{Notes on Nature Methods Brief Communication}
\begin{itemize}
\item Abstract: 3 sentences, 70 words.
\item Main text: 3 pages, 2 figures, 1000-1500 words, more figures possible if under 3 pages
\end{itemize}

\subsection*{Statistics on word count}
\detailtexcount
\newpage

%%%%%%%%%%%%%%%%%%%%%%%%%%%%%
% Supplementary Information %
%%%%%%%%%%%%%%%%%%%%%%%%%%%%%
\captionsetup*{format=largeformat}
\section{Something about something} \label{note:Note1} 
\Blindtext

%TC:endignore
%the command above ignores this section for word count

\end{document}
