\documentclass[times, twoside, watermark]{zHenriquesLab-StyleBioRxiv}
\usepackage{blindtext}
\usepackage{listings}

% Please give the surname of the lead author for the running footer
\leadauthor{Henriques} 

\begin{document}

\title{Discovering vulnerabilities in Python web applications}
\shorttitle{My Template}

% Use letters for affiliations, numbers to show equal authorship (if applicable) and to indicate the corresponding author
\author[1,\Letter]{83531 - Miguel Belém}
\author[2,\Letter]{87701 - Ricardo Ferreira}
\author[3,\Letter]{83576 - Vítor Nunes}

\maketitle

%TC:break Abstract
%the command above serves to have a word count for the abstract
\begin{abstract}
\blindtext
\end {abstract}
%TC:break main
%the command above serves to have a word count for the abstract

\begin{keywords}
bla | bla | bla | bla
\end{keywords}

\begin{corrauthor}

\end{corrauthor}

\section*{Implementation}
Our tool was develop in Python 3.

We used a AST object tree structure as an internal representation for the input program. Therefore
each instruction is converted to an object. We have objects for all major instructions like \textsc{If},
\textsc{Assignment}, \textsc{While}, \textsc{Variable}, \textsc{Compare}, etc... each one containing
only interesting values gathered from the AST json tree.
In the end the idea is to construct one single object, called \textsc{Root} which contains a block of instructions.
Intuitively, each instruction could contain others. An \textsc{If} contains a condition, a body of instructions
in case of conditions evaluates to True, and a else body of instructions, otherwise. A slice is then represented
as a matryoska of objects. We opted for this representation due to its flexibility. This is
accomplish by a recursive function that constructs the objects. It could be possible to use an iterative
function instead, however we would loose flexibility and we would increase code complexity.

We represent vulnerabilities, that we pretend to discover, as objects as well. We parsed them from the 
second JSON argument to an object. If the file has three vulnerabilities, for instance, we would create
three objects. We search for vulnerabilities one by one, this allows to detect and resolve ambiguities.
For example, suppose that we have two vulnerabilities when the same function name X, has different types.
In the first vulnerability X is a sink, and on the second X is a source. This could lead to ambiguities.
We then opt for searching one vulnerability at time. Unfortunately we did not have time but we pretend
to optimize allowing to search all vulnerabilities at once.

Since an uninitialized variable is considered a source, we need a way to identify which variables were
initialized. We built a table, designed by Symbol Table (Abr: SymTable) which saves every variable found
during the parsing and object construction.
Whenever we find a variable, we checked if it already exists on SymTable. If it was new we contruct a
new \textsc{Variable} object and added it to the SymTable. Otherwise SymTable just return the object.
This allows to have only one object per variable and use it multiple times by its reference.

\subsection*{Explicit Flows}
%explain how we use a symtable
In order to detect explicit flow we need a way to link sources and sinks through assigments.
We know that only variables can be sources and function calls can be either sources, sanitizers or sinks.
Then every variable and function call has now a type, which can be one of the three. It is now trivial
to identify if a given function call is a source, for instance, because we can check its type which is a
property of \textsc{FunctionCall} objects.

However if we assign a source \textsc{FunctionCall} F to a \textsc{Variable} A, then A it is not a source
but has a source which is F. We need a way to propagate this information back to A.
We design a change to our \textsc{Variable} and \textsc{FunctionCall} objects, which saves a list of source,
sanitizers and sinks. For instance, suppose the following example:

\begin{lstlisting}[language=Python]
a = 2
a = source()
\end{lstlisting}

After the first assigment, the sources (denoted by $Src_X$) of A will be empty.
After the second assigment, the sources of A will be the sources of the right value of the assigment.
$X = Y \rightarrow Src_X = Src_Y$

The object \textsc{FunctionCall} \textit{source} has it self as a source. $Src_{source} = source$
Then it is trivial to see that sources of \textit{a} will be the sources of \textit{source} $Src_a = source$

Now consider the case where we have a non source function, say \textit{func} which as several uninitialized
variables as arguments:

\begin{lstlisting}[language=Python]
a = func(b, c, d)
\end{lstlisting}

The sources of \textit{func} will be the union of sources of arguments in that function.
Intuitively, $Src_{func(arg_1, arg_2, ..., arg_n)} = Src_{arg_1} \cup Src_{arg_2} \cup ... \cup Src_{arg_n} = \bigcup_{arg_i}^{arg_n} Src_{arg_i} $
This applies for every other object structures, like \textsc{BinaryOperation}, etc and the logic is the
same for sanitizers.

Now consider an example where there are multiple branches:

\begin{lstlisting}[language=Python]
a = source()
if True:
    a = source2()
else:
    a = source3()

sink(a)
\end{lstlisting}

Whenever we detect a new block, which happens when there are indentation, we create a copy of the SymTable
to the block, so we can later merge changes.

For example, previous examples contains two blocks, one for the if body and another one for the else body.
After the \textsc{If} instruction, $\mathcal{S} = <a>$ $Src_a = source$
After the \textsc{If} instruction both if and else block SymTable will have $a$ but the source is different.
$If: Src_a = source2$ $Else: Src_a = source3$
Then we union \textsc{If} with \textsc{Else} $Src_a = source2, source3$ to merge multiple branches.
Next, we replace the current SymTable with the result of the previous union operation. $Src_a = source$ become
$Src_a = source2, source3$.

If \textsc{Else} body is ommited we perform the exactly same logic. \textsc{Else} receives a copy of the SymTable
which will not change since there are no instructions in that block. When merge occur, it will merge the \textsc{If}
SymTable with a copy of the current context.

\subsection*{Implicit Flows}
%explain what we do to check implicit flows, like propagate the sources of a condition to every assigment
%in the body


\section*{Imprecisions and Limitiations}

\section*{Challenges}

\section*{Bonus}

\section*{Testing}

\section*{Conclusions}

blablaba \ref{fig:computerNo} 
\blindtext

\section*{Bibliography}
\bibliography{zHenriquesLab-Mendeley}

%% You can use these special %TC: tags to ignore certain parts of the text.
%TC:ignore
%the command above ignores this section for word count
\onecolumn
\newpage

\section*{Word Counts}
This section is \textit{not} included in the word count. 
\subsection*{Notes on Nature Methods Brief Communication}
\begin{itemize}
\item Abstract: 3 sentences, 70 words.
\item Main text: 3 pages, 2 figures, 1000-1500 words, more figures possible if under 3 pages
\end{itemize}

\subsection*{Statistics on word count}
\detailtexcount
\newpage

%%%%%%%%%%%%%%%%%%%%%%%%%%%%%
% Supplementary Information %
%%%%%%%%%%%%%%%%%%%%%%%%%%%%%
\captionsetup*{format=largeformat}
\section{Something about something} \label{note:Note1} 
\Blindtext

%TC:endignore
%the command above ignores this section for word count

\end{document}
